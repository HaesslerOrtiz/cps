\documentclass[12pt, a4paper]{article}

% ============================
% PAQUETES
% ============================
\usepackage[utf8]{inputenc}
\usepackage[T1]{fontenc}
\usepackage[spanish]{babel}
\usepackage{graphicx}
\usepackage{subcaption}
\usepackage{setspace}
\usepackage{csquotes}
\usepackage{geometry}
\geometry{margin=2.5cm}
\usepackage{hyperref}
\hypersetup{
    colorlinks=true,
    linkcolor=black,
    citecolor=black,
    urlcolor=blue
}
\usepackage{cleveref}
\usepackage{booktabs}

% --- Para tablas que ajusten al ancho de página ---
\usepackage{tabularx}     % Columnas de ancho flexible
\usepackage{array}        % Nuevos tipos de columna
\usepackage{ragged2e}     % \RaggedRight en celdas
\setlength{\tabcolsep}{6pt}  % Espaciado horizontal entre columnas
\renewcommand{\arraystretch}{1.15} % Altura extra de filas

% Tipos de columna: L{<ancho>} y Y (flexible)
\newcolumntype{L}[1]{>{\RaggedRight\arraybackslash}p{#1}}
\newcolumntype{Y}{>{\RaggedRight\arraybackslash}X}

% Configuración de cleveref
\crefname{figure}{figura}{figuras}
\Crefname{figure}{Figura}{Figuras}
\crefname{subfigure}{figura}{figuras}
\Crefname{subfigure}{Figura}{Figuras}

% Conjunciones en español para cleveref
\AtBeginDocument{\renewcommand{\creflastconjunction}{ y~}}
\AtBeginDocument{\renewcommand{\crefrangeconjunction}{ a~}}
\AtBeginDocument{\renewcommand{\crefmiddleconjunction}{, }}

% Carpeta predeterminada para imágenes
\graphicspath{{resultados/}}

% ============================
% TÍTULO Y AUTOR
% ============================
\title{\textbf{Integración y configuración de análisis estático con SonarCloud, Semgrep y Snyk}}
\author{Haessler Joan Ortiz Moncada \\[0.5cm]
        Universidad Distrital Francisco José de Caldas \\
        Facultad de Ingeniería \\
        Curso: Construcción de Pruebas de Software}
\date{\today}

\begin{document}
\maketitle

% ============================
% INTRODUCCIÓN
% ============================
\section{Introducción}

En este informe documento el proceso de integración y configuración de tres plataformas para realizar 
análisis estático de código en el repositorio de mi autoría: \textbf{confi\_seg}.  

El objetivo principal es consolidar los resultados de seguridad y calidad del código, 
automatizando su ejecución mediante un único workflow en \texttt{GitHub Actions}.  

\subsection*{Nota:}
El archivo unificado \texttt{security-ci.yml}, que permite ejecutar las tres pruebas de forma coordinada, 
fue generado con la ayuda de una \textbf{Inteligencia Artificial}, en este caso ChatGPT, 
para optimizar la integración y reducir errores en la configuración.

% ============================
% METODOLOGÍA
% ============================
\section{Metodología}
Primero exploré cada plataforma para comprender sus características, los tipos de análisis que realizan y su 
integración con GitHub. Posteriormente, configuré un único workflow llamado \texttt{security-ci.yml} para ejecutar 
las tres pruebas automáticamente en cada \textit{push} o \textit{pull request}.

% ============================
% RESULTADOS
% ============================
\section{Resultados}

\subsection{Resultados de Semgrep}
El análisis realizado con \textbf{Semgrep} detectó múltiples vulnerabilidades relacionadas con el manejo inseguro de 
datos, consultas SQL y configuración de CORS.

\subsubsection*{Resumen general de hallazgos}
\begin{table}[h!]
\footnotesize
\centering
\begin{tabularx}{\textwidth}{@{} L{3.6cm} L{2.2cm} L{2.2cm} Y @{}}
\toprule
\textbf{Categoría} & \textbf{Lenguaje} & \textbf{Severidad} & \textbf{Archivos afectados} \\
\midrule
SQL concatenado sin sanitización & Python & Baja & \texttt{projects.py} (líneas 140, 163, 179, 188, 196) \\
Uso de métodos inseguros en DOM & JavaScript & Baja & \texttt{principal.js} (línea 216) \\
Consultas SQL formateadas sin parámetros & Python & Baja & \texttt{projects.py} (mismas líneas que anteriores) \\
Etiqueta sin atributo de integridad & HTML & Baja & \texttt{principal.html} (línea 67) \\
Política CORS insegura & Python & Media & \texttt{main.py} (línea 14) \\
\bottomrule
\end{tabularx}
\end{table}

\subsubsection*{Principales vulnerabilidades detectadas}
\begin{itemize}
    \item \textbf{SQL concatenado sin sanitización} (\texttt{projects.py}):  
    Varias funciones concatenan datos directamente en consultas SQL, lo que podría permitir ataques de \textbf{inyección SQL}.
    \item \textbf{Métodos inseguros en el DOM} (\texttt{principal.js}):  
    Uso de \texttt{innerHTML} con datos controlados por el usuario, potencialmente expuesto a ataques \textbf{XSS}.
    \item \textbf{Política CORS insegura} (\texttt{main.py}):  
    Configuración de CORS que permite solicitudes desde cualquier origen, lo que supone un riesgo de seguridad.
\end{itemize}

%---------------------------------------
\subsection{Resultados de Snyk}
El análisis con \textbf{Snyk} me reportó hallazgos principalmente en el backend (Python), con foco en construcción 
insegura de SQL, manejo de rutas y uso de parámetros no validados en conexiones. A continuación dejo el resumen:

\paragraph{Resumen general de hallazgos}
\begin{table}[h!]\footnotesize\centering
\begin{tabularx}{\textwidth}{@{} L{3.5cm} L{2.1cm} L{2.4cm} Y @{}}
\toprule
\textbf{Categoría} & \textbf{Lenguaje} & \textbf{Severidad} & \textbf{Archivos afectados (ejemplos)} \\
\midrule
SQL Injection (CWE-89) & Python & Alta & \texttt{backend/routers/projects.py} (múltiples líneas: 136, 175, 184, 192, 304, 311, 328, 336, 362, 471, 480, 486, 502, 508, 512, 525, 554, 571, 577, 585, 599, 623, 676, 743, etc.) \\
Path Traversal (CWE-23) & Python & Alta & \texttt{backend/routers/projects.py} (725, 859); \texttt{qgis\_tools/generar\_proyecto\_qgis.py} (35) \\
SSRF (CWE-918) & Python & Media & \texttt{backend/routers/projects.py} (155, 169, 218, 288) \\
Revocación/roles con SQL dinámico & Python & Media & \texttt{backend/routers/projects.py} (806, 813) \\
\bottomrule
\end{tabularx}
\end{table}

\paragraph{Principales vulnerabilidades detectadas}
\begin{itemize}
  \item \textbf{SQL Injection}: Encontré múltiples consultas construidas con \texttt{f-strings} que interpolan datos 
  controlados por el usuario. Se deben generar consultas parametrizadas (\texttt{cur.execute(sql, params)}), validación 
  estricta y, si es posible, ORM.
  \item \textbf{Path Traversal}: Se generan rutas a partir de entrada externa (lectura/escritura/borrado). Se deben  
  generar listas blancas de rutas, normalización con \texttt{os.path.abspath}, validaciones de nombre y extensión, y \texttt{Pathlib}.
  \item \textbf{SSRF}: Parámetros externos terminan en funciones de conexión; debo validar host, puerto, 
  nombre de DB y bloquear destinos internos.
  \item \textbf{Permisos y roles vía SQL dinámico}: Evitar interpolación directa al gestionar \texttt{GRANT/REVOKE/DROP ROLE}; 
  usar plantillas parametrizadas y catálogos del motor.
\end{itemize}

%---------------------------------------
\subsection{Resultados de SonarCloud}
Con \textbf{SonarCloud} obtuve una radiografía amplia de \textit{code smells}, bugs y vulnerabilidades en varias áreas del 
repositorio. Este es el resumen de los hallazgos:

\paragraph{Resumen general de hallazgos}
\begin{table}[h!]\footnotesize\centering
\begin{tabularx}{\textwidth}{@{} L{3.9cm} L{2.2cm} L{2.6cm} Y @{}}
\toprule
\textbf{Categoría} & \textbf{Lenguaje} & \textbf{Severidad} & \textbf{Archivos afectados (ejemplos)} \\
\midrule
Construcción de SQL desde datos del usuario & Python & Bloqueante/Crítica & \texttt{backend/routers/projects.py} (140, 196, 308, 315, 516, 575, 581, 615, 626, 661, 680, 747, etc.) \\
Complejidad cognitiva excesiva & Python/JS & Alta & \texttt{backend/routers/projects.py} (28, 30, 20, 26, 78); \texttt{frontend/assets/js/principal.js} (funciones con complejidad elevada) \\
Código inalcanzable / comentado / variables sin uso & Python/JS/HTML & Media/Baja & \texttt{projects.py} (unreachable y vars sin uso); \texttt{principal.js} (vars sin uso); \texttt{principal.html} (detalles de integridad/escape) \\
Convenciones de nombres y estilo & Python & Baja & \texttt{qgis\_tools/qgis\_core.py} (parámetros/campos que no cumplen regex recomendada) \\
\bottomrule
\end{tabularx}
\end{table}

\paragraph{Principales vulnerabilidades y problemas}
\begin{itemize}
  \item \textbf{SQL inseguro}: Igual que en Snyk, SonarCloud refuerza que debo eliminar la interpolación directa 
  en \texttt{execute(...)}, usar parámetros y/o ORM.
  \item \textbf{Complejidad cognitiva}: Varias funciones superan los umbrales; debo dividir responsabilidades, 
  extraer helpers y reducir condiciones anidadas.
  \item \textbf{Mantenibilidad}: Código inalcanzable, comentado y variables sin uso; limpieza y linters automáticos 
  (\texttt{ruff}/\texttt{flake8}/\texttt{eslint}) ayudarán.
  \item \textbf{Consistencia y convenciones}: Ajustar nombres a las guías de estilo (snake\_case en Python, etc.) 
  para mejorar legibilidad.
\end{itemize}

% ============================
% ANÁLISIS COMPARATIVO
% ============================
\section{Análisis comparativo}
Para comparar las tres plataformas, analicé sus principales características:

\begin{table}[h!]
\footnotesize
\centering
\begin{tabularx}{\textwidth}{@{} L{3.6cm} L{2.7cm} L{2.3cm} Y @{}}
\toprule
\textbf{Criterio} & \textbf{SonarCloud} & \textbf{Semgrep} & \textbf{Snyk} \\
\midrule
\textbf{Tipo de análisis} & Calidad, bugs y \textit{code smells} & SAST basado en reglas personalizadas & SCA (dependencias) + SAST \\
\textbf{Profundidad} & Alta (métricas y deuda técnica) & Media-Alta (reglas flexibles) & Alta en vulnerabilidades conocidas \\
\textbf{Lenguajes soportados} & +25 & +30 & +20 \\
\textbf{Integración con GitHub Actions} & Automática, genera \texttt{.yml} & Automática, genera \texttt{.yml} & Manual, requiere configuración propia \\
\textbf{Reportes} & Visuales y detallados & Consolidados y exportables & Incluye SARIF y dashboard propio \\
\textbf{Costo} & Gratis con límites & Gratis con límites & Gratis con límites\\
\bottomrule
\end{tabularx}
\end{table}

En general:
\begin{itemize}
    \item \textbf{SonarCloud} es ideal para evaluar calidad, métricas y deuda técnica.
    \item \textbf{Semgrep} es más flexible y permite crear reglas personalizadas adaptadas al proyecto.
    \item \textbf{Snyk} sobresale en la detección de vulnerabilidades en dependencias externas y propone soluciones rápidas.
\end{itemize}

% ============================
% CONCLUSIONES
% ============================
\section{Conclusiones}
\begin{enumerate}
    \item La combinación de \textbf{SonarCloud}, \textbf{Semgrep} y \textbf{Snyk} proporciona una cobertura completa del análisis estático de código, 
    otrogando un enfoque diferente y complementario.
    \item La automatización mediante un único workflow simplifica la gestión de resultados.
\end{enumerate}
\end{document}
